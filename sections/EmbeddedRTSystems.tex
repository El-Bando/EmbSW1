\section{Embedded RT Systems \buch{p.1}}
\subsection{Charakterisierung Embedded Systems \buch{p.5,p.6}}
Ein Embedded System \ldots
\begin{itemize}
  \item \ldots ist ein System, das einen Computer beinhaltet, aber keiner ist
  \item \ldots besteht üblicherweise aus HW und SW
  \item \ldots ist häufig ein Control System\\
\end{itemize}
Charakterisierung von Embedded Systems
  \begin{itemize}
    \item \textbf{Reaktive Systeme} : Interagieren mit ihrer Umgebung
    \item \textbf{Echtzeitsysteme/Real-time systems} : Definierbare zeitliche
    Anforderungen erfüllen
    \item \textbf{Verlässliche Systeme/Dependable systems}: Sehr hohe
    Zuverlässigkeitsanforderungen erfüllen
    \item \textbf{Weitere Anforderungen/Charakteristiken}: 
    \begin{itemize}
      \item Kleiner Energieverbrauch
      \item Kleine physikalische Abmessungen
      \item Lärm, Vibration, Feuchtigkeit etc\ldots
    \end{itemize}
  \end{itemize}

\subsubsection{Verfügbarkeit (Availability)}
Anteil an der Betriebsdauer, in  der das System seine Funktion
erfüllt:
%%%%%%%%%%%%%%%%%%%%%%%%%%%%%%%%%%%%%%%%%%%%%%%%%%%%%%%%%%%%%%%%%%%%%%%%%%%%%%%
\begin{equation}
Verfuegbarkeit = \frac{Gesamtzeit-Ausfallzeit}{Gesamtzeit}
\end{equation}

\subsubsection{Abstraktionsschichten}
Ein gutes Design beinhaltet unterschiedliche Abstraktionsschichten!\\
Mit Abstraktionsschichten \ldots
\begin{itemize}
  \item \ldots ist der Code lesbarer und besser auf eine andere Platform portierbar.
  \item \ldots wird professioneller Code für Hardware Abstraction Layer (HAL) erzeugt
  \item \ldots gibt es bei richtiger implementierung der HAL keine Geschwindigkeitseinbussen
\end{itemize}
%%%%%%%%%%%%%%%%%%%%%%%%%%%%%%%%%%%%%%%%%%%%%%%%%%%%%%%%%%%%%%%%%%%%%%%%%%%%%%%
\subsection{Definition Real-time Systems}
\textbf{Definition:} System, das Informationen innerhalb einer definierten Zeit
(deadline) bearbeiten muss. Es erfüllt explizite Anforderungen an die
Antwortzeiten.\\
\textbf{Antwortzeit} : Zeit vom Stimulus (Vorhandensein
Eingangswert) bis zum Erscheinen des Ausgangswert\\
\textbf{Fehlerhaftes System:} es werden nicht alle formal definierten Systemspezifikationen erfüllt.\\
\textbf{Soft real-time system}: System wird durch das verletzen von
Antwortzeiten nicht ernsthaft beeinflusst $\rightarrow$ Komforteinbussen (Bsp.:
Geldautomat)\\
\textbf{Firm real-time system}: Durch Verletzen weniger Antwortzeiten wird das
System nicht ernsthaft beeinflusst. Bei vielen Verletzungen jedoch $\rightarrow$
kompletter Ausfall, katastrophales Fehlverhalten (Bsp.: GPS-gesteuerter
Rasenmäher)\\
\textbf{Hard real-time system}: Durch Verletzen der Antwortzeiten wird das
System ernsthaft beeinflusst $\rightarrow$ kompletter Ausfall, katastrophales
Fehlverhalten (Bsp.: Helikoptersteuerung)\\
\textbf{Determinismus:} Für jeden möglichen Zustand und alle möglichen
Eingabewerte sind jederzeit der nächste Zustand und die Ausgabewerte definiert.
insbesondere race conditions k\"onnen dazu f\"uhren, dass der nächste Zustand davon abh\"angt. (Bsp. Ethernet Kollision)

\begin{multicols}{2}
  \textbf{Auslastung/Utilization}:
  Auslastung pro Prozess:
  \begin{center}
    \begin{tabular}{c c}
                                                   & $u_\text{i}$ = Auslastungsfaktor des Tasks i \\
      $u_\text{i} = \frac{e_\text{i}}{p_\text{i}}$ & $e_\text{i}$ = Exekutionszeit des Tasks i    \\
                                                   & $p_\text{i}$ = Executionperiode des Tasks i
    \end{tabular}\\
  \end{center}

\columnbreak

Für n periodische Tasks erhalten wir die gesamte Auslastung U: 
\begin{center}
\begin{equation}
U = \sum_{i=1}^{n}u_\text{i} = \sum_{i=1}^{n}\frac{e_\text{i}}{p_\text{i}}
\end{equation}
\end{center}
\end{multicols}

\begin{tabular}[c]{| l p{8cm} l |}
\hline\bfseries{Utilization(\%)} & \bfseries{Zone Type} & \bfseries{Typical Application}
\\\hline 0-25 & significant excess processing power-CPU may be more powerful than necessary & various
\\ 26-50 & very safe & various
\\ 51-68 & safe & various
\\\rowcolor{HSRBlue20} 69 & theoretical limit & embedded systems
\\ 70-82 & questionable & embedded systems
\\ 83-99 & dangerous & embedded systems
\\ 100+ & overload & stressed Systems
\\\hline
\end{tabular}
