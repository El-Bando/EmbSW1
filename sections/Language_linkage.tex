\section{language linkage}
Das \"ubersetzen von Programmiersprachen in einen Compiler-Syntax wird als language linkage bezeichnet.\\
Der Linker ben\"otigt einen eindeutigen Namen f\"ur jede Funktion. In C muss der Funktionsname 
Projektweit eindeutig sein. In C++ kann es komplizierter sein durch die Objektorierntierte Programmierung. 
Das Zuweisen eines eindeutigen Namens in C++ wird als \textit{Name Mangeling} bezeichnet.

\subsection{Interfacing with C}
\subsubsection{Plain Old Data Types (POD Types)}
\begin{itemize}
    \item Datentypen, die bereits in C vorhanden sind, werden als POD Types bezeichnet
    \item Dazu geh\"oren \dots
    \begin{itemize}
        \item \dots char, short, int, long, jeweils signed und unsigned
        \item \dots float, double
    \end{itemize}
    \item POD types funktionieren in C++ identisch zu C
\end{itemize}

\begin{multicols}{2}
    \subsubsection{C language linkage}
C Linker h\"angen f\"ur ihre interne Darstellung h\"aufig ein Underscore an den Funktionsnamen, 
um ein label in Assembler zu erhalten. Die Funktion foo() wird dann zu \_foo. 
Diese Form wird als \textit{C language linkage} bezeichnet.
    \vfill\null
    \columnbreak
    \subsubsection{C++ language linkage}
C++ Linker betreiben f\"ur ihre interne Darstellung \textit{Name Mangeling}. 
Die funktion \dots 
\begin{itemize}
    \item foo(int) wird zu \_foo\_i
    \item foo(double, int) wird zu \_foo\_d\_i
    \item MyClass::foo(int) wird zu \_MyClass\_foo\_i
\end{itemize}
Diese Form wird als \textit{C++ language linkage} bezeichnet. Mit \textit{c++filt} kann 
der unmangled name ermittelt werden.
\end{multicols}

\subsubsection{Probleme}
\begin{itemize}
    \item Annahme:\\Aus C++ wird eine Funktion foo(int) aufgerufen, die in einer compilierten
    C-Bibliothek vorliegt.
    \item Der \textcolor{red}{C++-Linker} sucht nach einer Funktion \_foo\textcolor{red}{\_i}
    \item Da es eine C-Bibliothek ist, heisst die Funktion aber \_foo
    \item Dem C++-Linker muss mitgeteilt werden, dass diese Funktion C Linkage hat.
\end{itemize}

\subsubsection{Festlegen der Language Linkage}
Im Funktionsprototypen (C++-seitig) kann die language linkage festgelegt werden.
\begin{lstlisting}
    extern "C" void foo(int); // use C language linkage
    extern void goo(int); // use C++ language linkage
    void hoo(int); // use C++ language linkage
    extern "C++" void koo(int); // use C++ language linkage
    extern "C"
    {
    // list multiple prototypes with C linkage, or
    // #include C header file(s)
    }
\end{lstlisting}
\newpage

\subsubsection{Anwendung extern 'C' im Headerfile}
\begin{multicols}{2}
\begin{lstlisting}
    #ifdef __cplusplus
    extern "C"
    {
    #endif
    // list multiple prototypes with C linkage, or
    // #include C header file(s)
    #ifdef __cplusplus
    }
    #endif
\end{lstlisting}
\vfill\null
\columnbreak
\_cplusplus ist dann definiert, wenn mit einem C++-Compiler compiliert wird.
Auf diese Weise versehene Headerfiles können sowohl in C‐ als auch in C++‐Dateien included 
werden. Beide Compiler (C, C++) k\"onnen diese Header \"ubersetzen.
\end{multicols}

\subsubsection{C++-Code aus C aufrufen}
\begin{itemize}
    \item C kennt keine Klassen
    \item C kennt den this-Pointer nicht
    \item Aus C können keine C++-Elementfunktionen direkt aufgerufen werden
    \item L\"osung: C Wrapper Funktionen f\_ur C zur Verf\"ugung stellen
    \begin{itemize}
        \item Erkl\"arung anhand konkretem Beispiel (Dining Philosopher)
    \end{itemize}
    \item Ein Client ruft folgende Funktion auf: \lstinline{void Philosopher::live()}
    \begin{itemize}
        \item Diese Funktion soll im Hintergrund einen eigenen Thread starten
    \end{itemize}
    \item Die Threadfunktion \lstinline{void Philosopher::lifeThread()} soll eine Elementfunktion der Klasse sein.
    \begin{itemize}
        \item Dadurch hat diese Funktion alle Vorteile einer Elementfunktion, z.B. Zugriff auf alle Klassenelemente
        \item Sie kann nun in C++ programmiert werden
    \end{itemize}
    \item Gestartet wird der Thread mit der \textbf{C-Funktion} pthread\_create()
\end{itemize}

\subsection{Programming style guide}
Programmierkonventionen erleichtern einen konsistenten Stil, der auch von anderen verstanden wird.
Grundsatz: \textit{Programme werden f\"ur Programmierer geschrieben, nicht f\"ur Compiler}
Beispiel Style Guide: Google C++ Style Guide

\subsubsection{Grunds\"atze}
\begin{itemize}
    \item Unwichtiges nicht \"uberbewerten
    \item Entdeckte Warnings vom Compiler muss nicht in den Codierrichtlinien erwähnt werden.
    \textcolor{red}{Mit h\"ochstem Warning level compilieren}
    \item Eine Anweisung pro Zeile
    \item Einr\"ucken
    \item \textbf{Konsistenz und Konsequenz}
    \item Namenskonventionen
    \begin{itemize}
        \item CamelCase
        \item Keine \_ im Namen
        \item Keine Pr\"afixe
        \item Anfangsbuchstaben von Variablen, Objekte, Funktionen und Methoden \textit{klein}
        \item Anfangsbuchstaben von eigenen Typen und Klassen \textit{gross}
        \item Makros ausschliesslich \textit{gross} schreiben
    \end{itemize}
\end{itemize}